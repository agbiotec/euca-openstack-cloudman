%% BioMed_Central_Tex_Template_v1.06
%%                                      %
%  bmc_article.tex            ver: 1.06 %
%                                       %

%%IMPORTANT: do not delete the first line of this template
%%It must be present to enable the BMC Submission system to 
%%recognise this template!!

%%%%%%%%%%%%%%%%%%%%%%%%%%%%%%%%%%%%%%%%%
%%                                     %%
%%  LaTeX template for BioMed Central  %%
%%     journal article submissions     %%
%%                                     %%
%%         <14 August 2007>            %%
%%                                     %%
%%                                     %%
%% Uses:                               %%
%% cite.sty, url.sty, bmc_article.cls  %%
%% ifthen.sty. multicol.sty		   %%
%%				      	   %%
%%                                     %%
%%%%%%%%%%%%%%%%%%%%%%%%%%%%%%%%%%%%%%%%%


%%%%%%%%%%%%%%%%%%%%%%%%%%%%%%%%%%%%%%%%%%%%%%%%%%%%%%%%%%%%%%%%%%%%%
%%                                                                 %%	
%% For instructions on how to fill out this Tex template           %%
%% document please refer to Readme.pdf and the instructions for    %%
%% authors page on the biomed central website                      %%
%% http://www.biomedcentral.com/info/authors/                      %%
%%                                                                 %%
%% Please do not use \input{...} to include other tex files.       %%
%% Submit your LaTeX manuscript as one .tex document.              %%
%%                                                                 %%
%% All additional figures and files should be attached             %%
%% separately and not embedded in the \TeX\ document itself.       %%
%%                                                                 %%
%% BioMed Central currently use the MikTex distribution of         %%
%% TeX for Windows) of TeX and LaTeX.  This is available from      %%
%% http://www.miktex.org                                           %%
%%                                                                 %%
%%%%%%%%%%%%%%%%%%%%%%%%%%%%%%%%%%%%%%%%%%%%%%%%%%%%%%%%%%%%%%%%%%%%%


\NeedsTeXFormat{LaTeX2e}[1995/12/01]
\documentclass[10pt]{bmc_article}    



% Load packages
\usepackage{cite} % Make references as [1-4], not [1,2,3,4]
\usepackage{url}  % Formatting web addresses  
\usepackage{ifthen}  % Conditional 
\usepackage{multicol}   %Columns
\usepackage[utf8]{inputenc} %unicode support
%\usepackage[applemac]{inputenc} %applemac support if unicode package fails
%\usepackage[latin1]{inputenc} %UNIX support if unicode package fails
\urlstyle{rm}
 
 
%%%%%%%%%%%%%%%%%%%%%%%%%%%%%%%%%%%%%%%%%%%%%%%%%	
%%                                             %%
%%  If you wish to display your graphics for   %%
%%  your own use using includegraphic or       %%
%%  includegraphics, then comment out the      %%
%%  following two lines of code.               %%   
%%  NB: These line *must* be included when     %%
%%  submitting to BMC.                         %% 
%%  All figure files must be submitted as      %%
%%  separate graphics through the BMC          %%
%%  submission process, not included in the    %% 
%%  submitted article.                         %% 
%%                                             %%
%%%%%%%%%%%%%%%%%%%%%%%%%%%%%%%%%%%%%%%%%%%%%%%%%                     


\def\includegraphic{}
\def\includegraphics{}



\setlength{\topmargin}{0.0cm}
\setlength{\textheight}{21.5cm}
\setlength{\oddsidemargin}{0cm} 
\setlength{\textwidth}{16.5cm}
\setlength{\columnsep}{0.6cm}

\newboolean{publ}

%%%%%%%%%%%%%%%%%%%%%%%%%%%%%%%%%%%%%%%%%%%%%%%%%%
%%                                              %%
%% You may change the following style settings  %%
%% Should you wish to format your article       %%
%% in a publication style for printing out and  %%
%% sharing with colleagues, but ensure that     %%
%% before submitting to BMC that the style is   %%
%% returned to the Review style setting.        %%
%%                                              %%
%%%%%%%%%%%%%%%%%%%%%%%%%%%%%%%%%%%%%%%%%%%%%%%%%%
 

%Review style settings
%\newenvironment{bmcformat}{\begin{raggedright}\baselineskip20pt\sloppy\setboolean{publ}{false}}{\end{raggedright}\baselineskip20pt\sloppy}

%Publication style settings
%\newenvironment{bmcformat}{\fussy\setboolean{publ}{true}}{\fussy}

%New style setting
\newenvironment{bmcformat}{\baselineskip20pt\sloppy\setboolean{publ}{false}}{\baselineskip20pt\sloppy}

% Begin ...
\begin{document}
\begin{bmcformat}


%%%%%%%%%%%%%%%%%%%%%%%%%%%%%%%%%%%%%%%%%%%%%%
%%                                          %%
%% Enter the title of your article here     %%
%%                                          %%
%%%%%%%%%%%%%%%%%%%%%%%%%%%%%%%%%%%%%%%%%%%%%%

\title{Interoperability of Commercial and Open-Source Clouds for Deployment of Bioinformatics Grid Computing Infrastructures with  CloudBioLinux-CloudMan}
 
%%%%%%%%%%%%%%%%%%%%%%%%%%%%%%%%%%%%%%%%%%%%%%
%%                                          %%
%% Enter the authors here                   %%
%%                                          %%
%% Ensure \and is entered between all but   %%
%% the last two authors. This will be       %%
%% replaced by a comma in the final article %%
%%                                          %%
%% Ensure there are no trailing spaces at   %% 
%% the ends of the lines                    %%     	
%%                                          %%
%%%%%%%%%%%%%%%%%%%%%%%%%%%%%%%%%%%%%%%%%%%%%%

\author{
             Alex Richter$^{1}$% 
             \and
             Konstantinos Krampis\correspondingauthor$^{2}$% 
             \email{Konstantinos Krampis\correspondingauthor - agbiotec@gmail.com}%
             \and
             Enis Afgan\correspondingauthor$^{3}$%
             \email{Enis Afgan\correspondingauthor - afgane@gmail.com}%
             \and
             Brad Chapman\correspondingauthor$^{4}$%
             \email{Brad Chapman\correspondingauthor - chapmanb@gmail.com}%
             }


%%%%%%%%%%%%%%%%%%%%%%%%%%%%%%%%%%%%%%%%%%%%%%
%%                                          %%
%% Enter the authors' addresses here        %%
%%                                          %%
%%%%%%%%%%%%%%%%%%%%%%%%%%%%%%%%%%%%%%%%%%%%%%

\address{ \\
\iid(1) Informatics Department, J. Craig Venter Institute, 10355 Science Center Dr., San Diego, CA 92121, USA \\
\iid(2) Informatics Department, J. Craig Venter Institute, 9704 Medical Center Dr., Rockville, MD 20850, USA  \\
\iid(3) Center for Informatics and Computing, Ruder Boskovic Institute, Zagreb, Croatia   \\
\iid(4) Dept. of Biostatistics, Harvard School of Public Health, 655 Huntington Avenue, Boston, MA, 02115, USA  \\
}

\maketitle


%%%%%%%%%%%%%%%%%%%%%%%%%%%%%%%%%%%%%%%%%%%%%%
%%                                          %%
%% The Abstract begins here                 %%
%%                                          %%  
%% Please refer to the Instructions for     %%
%% authors on http://www.biomedcentral.com  %%
%% and include the section headings         %%
%% accordingly for your article type.       %%   
%%                                          %%
%%%%%%%%%%%%%%%%%%%%%%%%%%%%%%%%%%%%%%%%%%%%%%


\begin{abstract}


\paragraph*{Background:} Conversion of self-contained binary Virtual Machine (VM) files that encapsulate the operating system, 
bioinformatic  tools and software libraries, is straightforward using utilities available by vendors of open-source 
and commercial clouds. Independent parallel processing of  datasets with bioinformatics pipelines running 
separately within multiple instances of a specific VM is an efficient approach, based on results presented in the current study. 
Nonetheless, virtualization and replication within clouds of informatics infrastructures such grid computing, has significant merit 
given that a majority of current bioinformatics pipelines need significant code refactoring to be used outside of institutional grids.

\paragraph*{Results:} In the present study, we use as foundation the CloudMan framework that allows bootstrapping of virtualized 
Sun Grid Engine (SGE) infrastructures using VMs on the Amazon EC2 cloud. Within the specific context of Cloudman and its interactions
with the Application Programming Interface (API) of the commercial Amazon EC2 or the open-source Eucalyptus and Openstack clouds
for the purpose of instantiating virtualized grids, we first examine cross-compatibility across these clouds. Cloudman is build using
components of web programming libraries that provide the interaction with the cloud API, and in our results we present special cases
observed for each cloud we examined, while also argue on the feasibility for development of similar frameworks that allow virtualization
of current institutional bioinformatics infrastructures and ease of deployment across the different cloud platforms. Finally, we present 
some test cases on performance and efficient utilization of underlying physical server resources for data analysis pipelines utilizing this
framework. 


\paragraph*{Conclusions:} Unlike ground up development of web applications that are designed around based on the cloud APIs and
to natively run within the cloud infrastructure, complex bioinformatics applications that are in advanced development state can seamlessly
encapsulated within VMs on cloud platforms. Nonetheless, for taking advantage if the scalability offered by the cloud, frameworks
like Cloudman that stand up grids or other commonly used approaches on institutional clusters are required, while also assuring that
the framework interoperates across the different cloud platforms.

\end{abstract}



\ifthenelse{\boolean{publ}}{\begin{multicols}{2}}{}







%%%%%%%%%%%%%%%%
%% Background %%
%%
\section*{Background}
\subsection*{Current State of Cloud Computing For Bioinformatics}

In recent years, cloud based bioinformatics data analysis systems such as Galaxy \cite{Goecks2010}, 
CloVR \cite{Angiuoli2011}, Cloud BioLinux \cite{Krampis2012} and BioKepler \cite{Altintas2011} were
released, allowing smaller laboratories and institutions to perform large-scale data analysis with genomic 
datasets. All these platforms are available on the Amazon Elastic Compute (EC2) cloud \cite{awsec2} , which 
provides data centers in US East and West regions, European Union, Asia and Australia \cite{ec2regions}, 
allowing any researcher worldwide  to connect to the cloud end-point closest to their geographic boundaries.  \pb

The Amazon cloud allows users to run Virtual Machines (VM) with various capacities in terms of the underlying
physical compute resources allocated to the VM, and the usage is billed on an hourly basis \cite{ec2price}. 
A VM is essentially a full-featured compute server, with virtual processors, memory and storage capacity, that runs 
on a Virtualization layer set on top of the hardware architecture on Amazon EC2. Furthermore, the operating system 
and all software components  are  encapsulated  within the VM, and data analysis  pipelines, databases, website portals  
and all their required code libraries can be distributed through a VM pre-configured and ready to execute. 
This approach for distributing software can remove many of the technical roadblocks for utilizing bioinformatics tools
developed at various research institutions, and consequently make the software easily accessible to the research community. \pb

Private clouds and Virtualbox where the VMs can run. \pb

Cloud BioLinux and Fabric and CloudMan. \pb


\subsection*{Open-Source Cloud Computing PlatForms As Private Clouds} 

\subsubsection*{Architecture of Eucalyptus Clouds and Installation at J. Craig Venter Institute}

The Eucalyptus open-source cloud \cite{euca} offers a fully compatible Application Programming Interface (API) with Amazon’s EC2 cloud, 
including the Simple Storage  Service (S3) \cite{s3},  Elastic Block Store (EBS) \cite{EBS} and additional services such as Auto Scaling and 
CloudWatch. The Eucalyptus API providing access to the cloud services is installed and runs on a Cloud Controller (CLC) physical machine,
and while Eucalyptus-specific tools are available for interacting with the API \cite{euca2ools}, the Amazon tools \cite{aws2ools} are fully 
compatible according to our experience. The CLC is essentially a software layer that besides serving the API and allowing interaction with 
the cloud, also tracks at a high level the status of physical and virtualized computing resources, in addition to being the coordinating service 
for the other Eucalyptus layers that constitute the cloud running on the physical cluster. For large cloud installations, more than one CLC
instances can be in place across different physical machines, ensuring scalability and load balancing when a large number of users 
interact with the cloud API, while also safeguarding the cloud operations in case of hardware failure.

In more detail, one of the cloud's software layers providing information to the CLC is the Cluster Controller (CC). The CLC delegates to the CC  
the task of gathering statistics from the physical nodes of the compute cluster running the Virtual Machines (VM), such as available disk space, 
number of VM running on each physical node, and compute load posed to the nodes.  In large cloud installations, similar to the CLC multiple 
CC instances can be run on different physical machines for fail protection and to balance the load of keeping track of hundreds or even 
thousand of VMs concurrently being started, terminated and utilized at large capacity by users. The cloud installation at  the J. Craig Venter 
Institute (JCVI) was comprised by four identical physical compute servers (16 core, 32GB memory, 1TB disk each), and one was used as the cloud's
head node to run a single instances of the CLC and CC components.

Regarding the cloud components that the users can access storage and computational resources from, the Amazon S3 storage corresponding 
service on Eucalyptus is called Walrus.  The Walrus and S3 storage model deviates from the typical file system of disk drives, it instead uses 
data objects that can be accessed by internet-identifiable unique Uniform Resource Location (URL) . 
A file with public permissions on a Walrus service for a cloud that is accessible outside of an institutional firewall, can be accessed by simply 
pasting its URL on a web browser, but for upload, modification, or for bulk data operations or for files that require authentication, one of the 
two Application Programming Interfaces (API) must be used to  access the service: the first is based on the Representational State Transfer 
protocol (REST, \cite{fielding2000})  and the second on the Simple Object Access Protocol (SOAP, \cite{soap}).  Alternatively, users can interact 
with the service through various desktop client applications  that have graphical front-ends, available for all different operating systems\cite{cyberduck, s3browse,s3fox}. \pb


Due to its front-facing role in serving data objects through identifiable URLs over the internet, Eucalyptus is designed to run the Walrus
component from the same physical node of the CLC, which provide access to the cloud's API for the user's applications. Walrus can be 
from a virtual machine instance running inside the cloud, in the case a user application is accessing the data objects via their URLs. At 
JCVI's cloud, we run the Walrus service on the same physical compute server as CLC and CC, the simplest configuration available for 
Eucalyptus given the small size (four server nodes) of our cluster. According to recommendations on the  Eucalyptus documentation \cite{eucadocs},
for larger cloud instances Walrus directory should be a mount of a Network Attached Storage (NAS, ref) system, allowing scalability for the 
amount of storage and performance in high user loads and data object requests. 

Another role of the Walrus is to serve as the VM image store, where compressed splits of binary files composing the VM template are stored, 
copied and joined at the physical nodes that will run the VM during boot time. In our experience with the small Eucalyptus cloud where a NAS 
was not used but rather we run Walrus within the limitations of the a single disk drive on a physical server, copying binary splits of the VM 
template to the physical nodes  during VM instance bootup, failed for VM larger than 100GB in size. We have found that using the caching 
mechanism available in Eucalyptus that places a VM template from Walrus on the CC \cite{eucacache} allowed us to boot VMs of size 200GB
and up within a few minutes.

The Node Controller (NC) component of Eucalyptus, is the cloud layer that boots and runs the VMs on the physical compute servers of the 
cluster, and constantly provides information to the CC for the available compute resources and the number of VMs on the server. At JCVI we 
have configured our Eucalyptus cloud with 4 NCs and therefore we essentially provide a master physical server running all cloud 
components (CLC, CC, Walrus, NC), and 3 worker servers running a single NC each dedicated to controlling resources where the VMs are running. 

The workflow  among the different components discussed so far, followed within a Eucalyptus cloud setup to boot a VM is as following: using the 
command line tools available either by Eucalyptus or Amazon or any of the available client applications, a call is made to the cloud API through
the CLC for booting a VM with specified computational capacity; the CLC checks its registry for the availability of resources, and routes
the request to a specific CC (if more than one CC are available), that has under its management a set of NC with available resources to
run the VM with requested capacity; the CC checks whether the VM is cached locally and if not a copy is initiated from Walrus and passed
over to the NC, while an unique IP address is assigned to the VM; finally, once the NC boots the VM its IP address and periodic updates
of its status are communicated back through the same chain to the CLC and the user's application through the API.

The last component of Eucalyptus is the Storage Controller (SC), which corresponds to the Elastic Block Store (EBS) storage option on the 
Amazon cloud. This provides virtual disk drives with identical filesystems to physical drives (POSIX-compliant \cite{mathur2007new}),  multiples
of which can be attached to a running VM. While the Amazon EBS provides virtual disk drives up to 1 TeraByte (TB), on Eucalyptus the maximum
size can be set through a configuration parameter and also depends on the space available on the underlying physical hard drive on the SC machine. 
While in theory the sizeof a single virtual data volume can reach up to the number of GB to fill the physical underlying  hard drive, an upper
limit has not been tested with the Eucalyptus cloud system. In our cloud and for our needs we started 2 250GB 
and a few smaller ones without any performance problems.

(find more from the other paper)EBS volumes persist past VM termination and are commonly 
used to store persistent data. An EBS volume cannot be shared between VMs and can only be accessed within the 
same availability zone in which the VM is running. Users can create snapshots from EBS volumes. Snapshots are stored
in Walrus and made available across availability zones. Eucalyptus with SAN support lets you use your enterprise-grade 
SAN devices to host EBS storage within a Eucalyptus cloud. Snapshots are placed on Walrus providing backup and a template
to create multiple copies of the EBS volume. On Amazon S3 since it is replicated across 3 zones, distaster recovery (look at the other paper)

These data volumes can be also used  as the bootable file system of a running VM stored only on a single Amazon data center. The 
disadvantage is that the EBS is mounted from the SC to the NC, using only the memory and CPU of the physical servers running the NC,
and creating a bottleneck the SC unless is connected to a NAS.




This is available but has not been tested in our Eucalyptus cloud.
VMware Broker (Broker or VB) is an optional Eucalyptus component, which is available if you are a Eucalyptus subscriber. 
VMware Broker enables Eucalyptus to deploy virtual machines (VMs) on VMware infrastructure elements. VMware Broker 
mediates all interactions between the CC and VMware hypervisors (ESX/ESXi) either directly or through VMware vCenter.

\pb


\subsubsection*{OpenStack}
Enis and collaborators. Describe the OpenStack architecture and components (Nova, S3 etc) that it provides.
Example private cloud installations that use it. \pb


%%%%%%%%%%%%%%%%%%%%%%%%%%%%
%% Results and Discussion %%
%%

\section*{Results and Discussion}


\subsection*{Porting the CloudBioLinux-CloudMan Framework to Private Clouds} 

\subsubsection*{The CloudMan Implementation For Scalable Computing on The Cloud}

Also present CloudMan as an enabling framework for porting bioinformatics pipelines, with example pipelines
and other applications implemented to run in parallel using the framework. \pb

Enis, technical description of CloudMan as it stands for Amazon EC2. For example it uses two data volumes,
pulls the code on boot from S3, how does it work with snapshots etc. This will serve as leeway for when describing
how we mapped these components to the other platforms, see below. \pb

\subsubsection*{Porting CloudBioLinux-CloudMan to Eucalyptus}
Alex, technical description of changes to the CloudMan EC2 codebase to run on Eucalyptus. How components that
CloudMan uses on EC2 where mapped to Eucalyptus. Specific nagging that Eucalyptus does with running CloudMan -
i.e. different responses from the API returned when doing calls via boto, and changes in the cloudman codebase in
order to take care of these. Also is there something that sysadmins that set up Eucalyptus should pay attention too 
(in the settings, in order to leverage those applications that create scriptable infrastructures by calling the API) \pb

\subsubsection*{Porting CloudBioLinux-CloudMan to OpenStack}
Enis, technical description of changes to the CloudMan EC2 codebase to run on OpenStack.  Changes that have be
made to code in order to communicate with the API correctly. What sysadmins that set up OpenStack should pay attention 
too... \pb

\subsection*{Execution and Performance of Bioinformatic Pipelines on Private Clouds} 

\subsubsection*{An HMM-BLAST Pipeline for Metagenomic Annotation at JCVI}
Ntino... I got the numbers, need to write it. \pb

\subsubsection*{A Variant Calling Analysis Pipeline at HSPH}
\pb

\subsubsection*{Enis Other Example of Pipeline}
\pb

%%%%%%%%%%%%%%%%%%%%%%
\section*{Conclusions}  \ldots


\bigskip

%%%%%%%%%%%%%%%%%%%%%%%%%%%%%%%%
\section*{Author's contributions}
    Alex Richter and Konstantinos Krampis worked on the technical aspects of porting CloudMan
    to the Eucalyptus Cloud. Enis Afgan and .... worked on porting CloudMan on the Openstack
    Cloud. Brad Chapman implemented the parallel sequence data analysis pipelines used for benchmarking
    the two different Clouds.

    

%%%%%%%%%%%%%%%%%%%%%%%%%%%
\section*{Acknowledgements}
  \ifthenelse{\boolean{publ}}{\small}{}
  Text for this section \ldots
 
%%%%%%%%%%%%%%%%%%%%%%%%%%%%%%%%%%%%%%%%%%%%%%%%%%%%%%%%%%%%%
%%                  The Bibliography                       %%
%%                                                         %%              
%%  Bmc_article.bst  will be used to                       %%
%%  create a .BBL file for submission, which includes      %%
%%  XML structured for BMC.                                %%
%%  After submission of the .TEX file,                     %%
%%  you will be prompted to submit your .BBL file.         %%
%%                                                         %%
%%                                                         %%
%%  Note that the displayed Bibliography will not          %% 
%%  necessarily be rendered by Latex exactly as specified  %%
%%  in the online Instructions for Authors.                %% 
%%                                                         %%
%%%%%%%%%%%%%%%%%%%%%%%%%%%%%%%%%%%%%%%%%%%%%%%%%%%%%%%%%%%%%

\newpage
{\ifthenelse{\boolean{publ}}{\footnotesize}{\small}
 \bibliographystyle{bmc_article}  % Style BST file
  \bibliography{bmc_article} }     % Bibliography file (usually '*.bib' ) 

%%%%%%%%%%%

\ifthenelse{\boolean{publ}}{\end{multicols}}{}







\end{bmcformat}
\end{document}







