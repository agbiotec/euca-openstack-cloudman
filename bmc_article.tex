%% BioMed_Central_Tex_Template_v1.06
%%                                      %
%  bmc_article.tex            ver: 1.06 %
%                                       %

%%IMPORTANT: do not delete the first line of this template
%%It must be present to enable the BMC Submission system to 
%%recognise this template!!

%%%%%%%%%%%%%%%%%%%%%%%%%%%%%%%%%%%%%%%%%
%%                                     %%
%%  LaTeX template for BioMed Central  %%
%%     journal article submissions     %%
%%                                     %%
%%         <14 August 2007>            %%
%%                                     %%
%%                                     %%
%% Uses:                               %%
%% cite.sty, url.sty, bmc_article.cls  %%
%% ifthen.sty. multicol.sty		   %%
%%				      	   %%
%%                                     %%
%%%%%%%%%%%%%%%%%%%%%%%%%%%%%%%%%%%%%%%%%


%%%%%%%%%%%%%%%%%%%%%%%%%%%%%%%%%%%%%%%%%%%%%%%%%%%%%%%%%%%%%%%%%%%%%
%%                                                                 %%	
%% For instructions on how to fill out this Tex template           %%
%% document please refer to Readme.pdf and the instructions for    %%
%% authors page on the biomed central website                      %%
%% http://www.biomedcentral.com/info/authors/                      %%
%%                                                                 %%
%% Please do not use \input{...} to include other tex files.       %%
%% Submit your LaTeX manuscript as one .tex document.              %%
%%                                                                 %%
%% All additional figures and files should be attached             %%
%% separately and not embedded in the \TeX\ document itself.       %%
%%                                                                 %%
%% BioMed Central currently use the MikTex distribution of         %%
%% TeX for Windows) of TeX and LaTeX.  This is available from      %%
%% http://www.miktex.org                                           %%
%%                                                                 %%
%%%%%%%%%%%%%%%%%%%%%%%%%%%%%%%%%%%%%%%%%%%%%%%%%%%%%%%%%%%%%%%%%%%%%


\NeedsTeXFormat{LaTeX2e}[1995/12/01]
\documentclass[10pt]{bmc_article}    



% Load packages
\usepackage{cite} % Make references as [1-4], not [1,2,3,4]
\usepackage{url}  % Formatting web addresses  
\usepackage{ifthen}  % Conditional 
\usepackage{multicol}   %Columns
\usepackage[utf8]{inputenc} %unicode support
%\usepackage[applemac]{inputenc} %applemac support if unicode package fails
%\usepackage[latin1]{inputenc} %UNIX support if unicode package fails
\urlstyle{rm}
 
 
%%%%%%%%%%%%%%%%%%%%%%%%%%%%%%%%%%%%%%%%%%%%%%%%%	
%%                                             %%
%%  If you wish to display your graphics for   %%
%%  your own use using includegraphic or       %%
%%  includegraphics, then comment out the      %%
%%  following two lines of code.               %%   
%%  NB: These line *must* be included when     %%
%%  submitting to BMC.                         %% 
%%  All figure files must be submitted as      %%
%%  separate graphics through the BMC          %%
%%  submission process, not included in the    %% 
%%  submitted article.                         %% 
%%                                             %%
%%%%%%%%%%%%%%%%%%%%%%%%%%%%%%%%%%%%%%%%%%%%%%%%%                     


\def\includegraphic{}
\def\includegraphics{}



\setlength{\topmargin}{0.0cm}
\setlength{\textheight}{21.5cm}
\setlength{\oddsidemargin}{0cm} 
\setlength{\textwidth}{16.5cm}
\setlength{\columnsep}{0.6cm}

\newboolean{publ}

%%%%%%%%%%%%%%%%%%%%%%%%%%%%%%%%%%%%%%%%%%%%%%%%%%
%%                                              %%
%% You may change the following style settings  %%
%% Should you wish to format your article       %%
%% in a publication style for printing out and  %%
%% sharing with colleagues, but ensure that     %%
%% before submitting to BMC that the style is   %%
%% returned to the Review style setting.        %%
%%                                              %%
%%%%%%%%%%%%%%%%%%%%%%%%%%%%%%%%%%%%%%%%%%%%%%%%%%
 

%Review style settings
%\newenvironment{bmcformat}{\begin{raggedright}\baselineskip20pt\sloppy\setboolean{publ}{false}}{\end{raggedright}\baselineskip20pt\sloppy}

%Publication style settings
%\newenvironment{bmcformat}{\fussy\setboolean{publ}{true}}{\fussy}

%New style setting
\newenvironment{bmcformat}{\baselineskip20pt\sloppy\setboolean{publ}{false}}{\baselineskip20pt\sloppy}

% Begin ...
\begin{document}
\begin{bmcformat}


%%%%%%%%%%%%%%%%%%%%%%%%%%%%%%%%%%%%%%%%%%%%%%
%%                                          %%
%% Enter the title of your article here     %%
%%                                          %%
%%%%%%%%%%%%%%%%%%%%%%%%%%%%%%%%%%%%%%%%%%%%%%

\title{Interoperability of Commercial and Open-Source Clouds for Bioinformatics Grid Computing with CloudBioLinux-CloudMan}
 
%%%%%%%%%%%%%%%%%%%%%%%%%%%%%%%%%%%%%%%%%%%%%%
%%                                          %%
%% Enter the authors here                   %%
%%                                          %%
%% Ensure \and is entered between all but   %%
%% the last two authors. This will be       %%
%% replaced by a comma in the final article %%
%%                                          %%
%% Ensure there are no trailing spaces at   %% 
%% the ends of the lines                    %%     	
%%                                          %%
%%%%%%%%%%%%%%%%%%%%%%%%%%%%%%%%%%%%%%%%%%%%%%

\author{
             Alex Richter$^{1}$% 
             \and
             Konstantinos Krampis\correspondingauthor$^{2}$% 
             \email{Konstantinos Krampis\correspondingauthor - agbiotec@gmail.com}%
             \and
             Enis Afgan\correspondingauthor$^{3}$%
             \email{Enis Afgan\correspondingauthor - afgane@gmail.com}%
             \and
             Brad Chapman\correspondingauthor$^{4}$%
             \email{Brad Chapman\correspondingauthor - chapmanb@gmail.com}%
             }


%%%%%%%%%%%%%%%%%%%%%%%%%%%%%%%%%%%%%%%%%%%%%%
%%                                          %%
%% Enter the authors' addresses here        %%
%%                                          %%
%%%%%%%%%%%%%%%%%%%%%%%%%%%%%%%%%%%%%%%%%%%%%%

\address{ \\
\iid(1) Informatics Department, J. Craig Venter Institute, 10355 Science Center Dr., San Diego, CA 92121, USA \\
\iid(2) Informatics Department, J. Craig Venter Institute, 9704 Medical Center Dr., Rockville, MD 20850, USA  \\
\iid(3) Center for Informatics and Computing, Ruder Boskovic Institute, Zagreb, Croatia   \\
\iid(4) Dept. of Biostatistics, Harvard School of Public Health, 655 Huntington Avenue, Boston, MA, 02115, USA  \\
}

\maketitle


%%%%%%%%%%%%%%%%%%%%%%%%%%%%%%%%%%%%%%%%%%%%%%
%%                                          %%
%% The Abstract begins here                 %%
%%                                          %%  
%% Please refer to the Instructions for     %%
%% authors on http://www.biomedcentral.com  %%
%% and include the section headings         %%
%% accordingly for your article type.       %%   
%%                                          %%
%%%%%%%%%%%%%%%%%%%%%%%%%%%%%%%%%%%%%%%%%%%%%%


\begin{abstract}


\paragraph*{Background:} Porting across cloud platforms of complex bioinformatic pipelines that are self-contained within Virtual 
Machine (VM) files that encapsulate the operating system, bioinformatic  tools and software libraries, is straightforward using 
utilities available by vendors of open-source and commercial clouds. Parallel processing of large-scale datasets using bioinformatics 
pipelines running independently within multiple instances of a specific VM is a simple but efficient approach, with such evidence 
presented in the current study. Nonetheless, effort towards virtualization and replication within clouds of such grid computing found 
on institutional compute clusters has significant merit, given that a majority of current bioinformatics pipelines are tied to institutional grids 
and would need significant code refactoring to be used outside of such informatics infrastructures. 

\paragraph*{Results:} In the present study, we leverage the CloudMan framework that allows bootstrapping of virtualized 
Sun Grid Engine (SGE) infrastructures using VMs on the Amazon EC2 cloud. Within the specific context of Cloudman and its interactions
with the Application Programming Interface (API) of the commercial Amazon EC2 or the open-source Eucalyptus and Openstack clouds
for the purpose of instantiating virtualized grids, we first examine API cross-compatibility among these clouds. Furthermore, we argue 
on the feasibility for development of similar frameworks to CloudMan that allow virtualization of current institutional bioinformatics 
infrastructures and facilitate deployment  of current bioinformatic pipelines with minimal code refactoring across the different cloud 
platforms. Finally, we present test cases on performance and efficient utilization of underlying physical server resources for data analysis 
pipelines utilizing this framework. 


\paragraph*{Conclusions:} Unlike newly developed web applications that can be designed from the ground up to utilize the cloud 
APIs and natively run within the cloud infrastructure, most complex bioinformatics pipelines that are in advanced development state 
can only be encapsulated within VMs along with all their software dependencies. Nonetheless, for taking advantage of the scalability 
offered by the cloud, additional frameworks similar to Cloudman are required that stand up virtualized grids and emulate the most 
common infrastructures found on institutional clusters where usually most of the bioinformatics pipelines run.

\end{abstract}



\ifthenelse{\boolean{publ}}{\begin{multicols}{2}}{}







%%%%%%%%%%%%%%%%
%% Background %%
%%
\section*{Background}

\subsection*{Democratizing large-scale data analysis for next-gen sequencing}

The growth in sequencing throughput and reduction of sequencing cost has shifted the challenge from data 
generation to data storage and analysis \ref{Stein2010}. Considering only the field of genomics, tens of TeraBytes 
of data have been released in the recent five years \cite{Mason2012}. Acquiring the reads during a whole genome shotgun 
sequencing project is only the first step, and must be followed by bioinformatic analysis that involves running 
post-sequencing data analysis software, such as read quality checks, trimming and barcode deconvolution, in 
addition to assembly, annotation, and output interpretation in order to generate scientific value from the sequence 
data. \pb

Sequencing instruments are typically bundled with only minimal computational and storage capacity, that is sufficient 
for  data capture during runs. In recent years, a new paradigm in genomic sequencing has emerged 
with small-factor, benchtop sequencers, that provide enough throughput for sequencing small to medium-sized 
genomes, but are priced at levels  affordable for researchers running independent  laboratories. Examples include the 
GS Junior by 454, MiSeq by Illumina and Ion Proton by  Life Technologies, providing per run sequencing capacity at 
0.035Gb, 1.5Gb and 1Gb respectively (review in \cite{Loman2012a}). 
Nonetheless, given the scale of genomic datasets, scientific value cannot be obtained from an investment 
in a sequencer, unless it is accompanied by an equal investment in bioinformatics infrastructure. Processing large 
amounts of genomic data requires access to a range of technical capabilities including expertise in information 
technology (IT), bioinformatics, and software engineering, as well as to large-scale computational and storage 
capabilities. \pb

Centralized bioinformatic centers offering data analysis services to the community where expertise and computational 
resources are concentrated and economies of scale can minimize costs, have been established with support from 
federal grants in the US and Europe. Nonetheless, many of these centers are struggling to keep up with the volume and 
variety of data that is being produced, and in most cases analysis is restricted to sequencing data generated 
in-house, using genome samples submitted by community researchers. Offering a publicly accessibly analysis platform 
for 3rd party, independent  laboratories that perform genome sequencing, is available only with some exceptions 
at these centers and is limited in scope (for example MG-RAST for metagenomics \cite{Aziz2010}, \cite{Wilkening2009}). 
Compute resources of a national scale such as TeraGrid \cite{wilkins2008teragrid}  and XSEDE \cite{backmann2012xsede} 
provide large-scale computer clusters to the public, but support for bioinformatics software and installation of complex 
data analysis pipelines is limited. \pb

The adoption of cloud computing technologies is providing an opportunity to change this paradigm by effectively 
democratizing the field, meaning that individual researchers and labs can now have access to the resources that 
were previously only available to the large sequencing and bioinformatic centers. Nonetheless, a problem arises in 
regards to configuring and utilizing bioinformatics software for next-generation sequencing, where significant 
expertise is required on Unix operating systems, programming languages, software compilation from source code, 
file systems and large-scale data management. For smaller laboratories this can become a significant obstacle, as in 
addition to coming up with the funds for building an informatics infrastructure with capacity to handle the large 
computations, they also need to hire trained bioinformaticians competent to install, configure and run sequence 
analysis software tools and manage the data, which can present a higher expense than that of acquiring the hardware. \pb

Along these lines, virtualization technologies allow entire compute servers including the operating system and 
all the necessary software packages for data analysis, to be pre-installed, configured and ready-to-execute 
within a Virtual Machine (VM).  This approach allows for making open-source software requiring complex dependencies 
and installation procedures widely accessible to the research community. A VM is essentially an emulation of a compute 
server, with virtual processors, memory  and storage capacity, in the form of a single binary file that can be executed on
top of a virtualization layer running on a physical compute server\cite{Uhlig2005}. Virtualization technologies led to the 
development of cloud computing  services, where remote computer server farms can be rented on an hourly basis by 
researchers and used for scalable, on-demand computation. Cloud services offer high-performance computer hardware with a 
virtualization layer, upon which a user executes VM servers \cite{armbrust2010view}. Renting computational capacity from these 
services has the potential to eliminate many of the upfront capital and effort expenses of building technology 
infrastructure for genome sequencing informatics, and as result transform the analysis and data processing 
tasks into well defined operational costs.


\subsection*{Current State of Cloud Computing For Bioinformatics}

In recent years, cloud based bioinformatics data analysis systems such as Galaxy \cite{Goecks2010}, 
CloVR \cite{Angiuoli2011}, Cloud BioLinux \cite{Krampis2012} and BioKepler \cite{Altintas2011} were
released, allowing smaller laboratories and institutions to perform large-scale data analysis with genomic 
datasets. All these platforms are available on the Amazon Elastic Compute (EC2) cloud \cite{awsec2} , which 
provides data centers in US East and West regions, European Union, Asia and Australia \cite{ec2regions}, 
allowing any researcher worldwide  to connect to the cloud end-point closest to their geographic boundaries.  \pb

The Amazon cloud allows users to run Virtual Machines (VM) with various capacities in terms of the underlying
physical compute resources allocated to the VM, and the usage is billed on an hourly basis \cite{ec2price}. 
This particular Cloud service consists of thousands of computer servers with petabytes of storage, leveraging 
economies of scale to achieve low operational costs that in turn offers as savings to users. Furthermore, the Amazon Cloud 
has data centers in US East and West regions, European Union, Asia and Australia \cite{ec2regions}, providing researchers 
worldwide with the ability to tap into a large pool of computational resources, outside of institutional, 
economic or geographic boundaries. \pb

Researchers with access to a computing cluster at their home institution have the option to implement their own private Cloud 
platform and run VM servers build locally or downloaded from remote Clouds, using the open-source Eucalyptus \cite{euca} or 
OpenStack \cite{openstack} Clouds. OpenStack is the official Cloud  bundled with the Ubuntu Linux operating system \cite{ubuntucloud} 
and can be easily installed on clusters running other  Linux versions \cite{openstackother}, and similarly for Eucalyptus \cite{eucalyptusother}. 
Both of these Cloud  platforms offer identical Application Programming Interfaces (API) with Amazon EC2, and applications developed 
either on Eucalyptus or OpenStack work seamlessly on the Amazon Cloud and vice-versa. Furthermore, users have the option to 
copy and execute VM server snapshots across installations of these Clouds as it has been demonstrated for the Cloud BioLinux VM \cite{Krampis2012}.
For users who do not own a local compute cluster, neither have the available funds to lease computing time from 
Amazon, the government-funded Magellan cluster \cite{magellan} provides an OpenStack Cloud where they can apply 
for a  account. In addition, a number of academic computing centers in both the US and other countries have similar clusters 
with OpenStack installed \cite{openstackinstalls}, where scientists could request access, in addition to the  the Eucalyptus Community Cloud \cite{ecc}. 
Finally, an option for users is to to run the VM servers on a desktop  computer, using virtualization software such as VirtualBox \cite{vbox}, that is also 
open-source and can be installed in a single step on Windows, Mac or Linux  computers. The Cloud BioLinux project for example, 
provides  VM server snapshots that run on both private and Commercial clouds, and VirtualBox \cite{cblorg}. 
 \pb

We have recently expanded Cloud BioLinux by adding support for advanced users through a VM development framework for building and 
distributing customized bioinformatics virtual machines, adding to the community aspect of the project and enhancing the available 
features for collaborative bioinformatics VM development. For this purpose we implemented a modular software management system that 
simplifies the process of selecting the bioinformatics tools to be included in the VM, automates building of a new VM with the specified 
software, and seamlessly deploys across different cloud platforms. The overall goal is to offer a system for maintaining a range of specialized 
VM setups for serving different computing needs within the bioinformatics community, and allow researchers to focus on the next challenges 
of providing data, documentation, and the development of scalable analysis pipelines.  \pb


\subsection*{Open-Source Cloud Computing PlatForms As Private Clouds} 

\subsubsection*{Architecture of Eucalyptus Clouds and Installation at J. Craig Venter Institute}

The Eucalyptus open-source cloud \cite{euca} offers a fully compatible Application Programming Interface (API) with Amazon’s EC2 cloud, 
including the Simple Storage  Service (S3) \cite{s3},  Elastic Block Store (EBS) \cite{EBS} and additional services such as Auto Scaling and 
CloudWatch. The Eucalyptus API  is installed and runs on a Cloud Controller (CLC) physical machine, and both Eucalyptus-specific tools are available 
for interacting with the API \cite{euca2ools} while the Amazon tools \cite{aws2ools} are fully compatible as well. The CLC is essentially a software 
layer that besides serving the API and allowing interaction with the cloud, also tracks all data on the status of physical and virtualized computing resources, 
in addition to being the coordinator for the other Eucalyptus service that constitute the cloud. For large cloud installations, more than one CLC
instances can be in place across different physical machines, ensuring scalability and load balancing when a large number of users 
interact with the cloud API, while also safeguarding the cloud operations in case of hardware failure.

In more detail, one of the cloud's software layers providing information to the CLC is the Cluster Controller (CC). The CLC delegates to the CC  
the task of gathering statistics from the physical nodes of the compute cluster running the Virtual Machines (VM), such as available disk space, 
number of VM running on each physical node, and compute load.  In large cloud installations, similarly to the CLC multiple CC instances can run 
concurrently on different physical machines for fail protection and to balance the load of keeping track of tens of VMs being started, terminated
and accessed by a large number of users. The cloud installation at  the J. Craig Venter Institute (JCVI) is comprised by four identical physical compute 
servers (16 core, 32GB memory, 1TB disk each), and one was used as the cloud's head node to run a single instances of the CLC and CC components.

The Amazon S3 storage corresponding service on Eucalyptus is called Walrus, and both deviate from the typical file system of disk drives, as they 
instead use data objects that can be accessed by internet-identifiable unique Uniform Resource Location (URL). Specifically, a file with public permissions 
on a Walrus service for a cloud that resides outside of an institutional firewall, can be accessed by simply pasting its URL on a web browser, but for 
upload, modification, or for bulk data operations or for files that require authentication, one of the two Application Programming Interfaces (API) must 
be used to  access the service: the first is based on the Representational State Transfer protocol (REST, \cite{fielding2000}), while the second on the Simple 
Object Access Protocol (SOAP, \cite{soap}).  Alternatively, users can interact with the service through various desktop client applications that provide
a graphical front-end, available for all different operating systems\cite{cyberduck, s3browse,s3fox}. \pb

Due to its front-facing role in serving data objects through identifiable URLs over the internet, Eucalyptus is designed to run the Walrus
service from the same physical node of the CLC, which provides access to the cloud's API for the user's applications. Walrus can be also 
accessed from a VM running inside the cloud, in the case a user application is requesting data objects from the service via their URLs. At 
JCVI's cloud, we run the Walrus service on the same physical compute server as CLC and CC, the simplest configuration available for 
Eucalyptus given the small size (four server nodes) of our cluster. According to recommendations on the  Eucalyptus documentation \cite{eucadocs},
for larger cloud instances the Walrus physical storage should be connected a Network Attached Storage (NAS, ref) system, allowing scalability
during high user loads and data object requests. 

Another role of the Walrus is to serve as the VM image store, where compressed splits of binary files composing a VM template are stored, 
copied and joined at the physical node that boots the VM. In our experience with the small Eucalyptus cloud where a NAS  was not used but rather 
we run Walrus within the limitations of the a single disk drive on a physical server, and Eucalyptus copies binary splits of the VM 
template to the physical nodes during VM booting,  the process times out and fails for VM larger than 100GB in size. We have found that by using 
the caching mechanism available in Eucalyptus and placing a VM template from Walrus on the CC \cite{eucacache} allowed us to start VMs of size 200GB
and up within a few minutes.

The Node Controller (NC) component of Eucalyptus, is the cloud layer that boots and runs the VMs on the physical compute servers of the 
cluster, and constantly provides information to the CC on the available compute resources and number of VMs on the server. At JCVI we 
have configured our Eucalyptus cloud with 4 NCs, and therefore we essentially provided a master physical server running all cloud 
components (CLC, CC, Walrus, NC), and 3 worker servers running a single NC each. 

The workflow involving the different Eucalyptus cloud components discussed so far to boot a VM is as following: using the command line tools 
available either by Eucalyptus or Amazon or any of the available client applications, a call is made to the cloud API through the CLC for booting 
a VM with specified computational capacity (within the limits of the physical compute resources, and as allowed from the cloud's administrative
interface); the CLC checks its registry for the availability of resources, and routes the request to a specific CC (if more than one CC are available), 
that has under its management a set of NC with available resources to run the VM with requested capacity; the CC checks whether the VM is cached 
locally and if not a copy is initiated from Walrus and passed over to the NC, while an unique IP address is assigned to the VM; finally, once the NC 
boots the VM its IP address and periodic updates of its status are communicated back through the same chain to the CLC and the user's application 
through the API.

The last component of Eucalyptus is the Storage Controller (SC), which corresponds to the Elastic Block Store (EBS) storage option on the 
Amazon cloud. This provides virtual disk drives with identical filesystems to physical drives (POSIX-compliant \cite{mathur2007new}),  multiples
of which can be attached to a running VM. While the Amazon EBS provides virtual disk drives up to 1 TeraByte (TB), on Eucalyptus the maximum
size can be set through a configuration parameter and also depends on the space available on the underlying physical hard drive on the SC machine. 
To the best of our knowledge no upper limit to the sizes of the virtual drives has not been established for Eucalyptus cloud system, and in theory the 
size of a single virtual data volume can match the remaining capacity of the underlying physical hard drive. For JCVI's cloud where the SC had 
available a 1000GB drive, we have successfully started two EBS volumes at 250GB, one at 150GB and a few smaller ones without any performance problems.

The EBS volumes persist past VM termination and be re-attached to a new VM,  making them ideal for storing persistent data such as relational or 
bioinformatics databases. An already attached EBS volume cannot be shared between VMs, unless sharing is through a software layer such as the 
Network File System (NFS) is used on the VM that has the volume attached. Users can create snapshots from EBS volume, which are placed on the Walrus 
storage first providing backup, while also can be used as template images to create multiple identical copies of the original EBS volume. Furthermore, on 
the Amazon cloud, EBS snapshots are copied in triplicate across three data centers in different geographic zones of the S3 storage.  

EBS volumes can be also used as the main file system of a VM, allowing to use large-sized bootable VMs without the timeout issue we observed when using
using Walrus-backed VM templates. The disadvantage is that EBS volumes are placed on the SC that on the Eucalyptus cloud resides on a different physical
server than the  NC. Therefore, only the memory and CPU of the physical server running the NC service is utilized for the VM, while the EBS storage has to 
be attached to the VM over the network from the SC. This has the disadvantage of significantly increasing the network traffic within the cluster that runs the 
cloud, while creating an additional bottleneck based on the storage available on the SC. This approach does not leveraging the scalability offered with addition
of local storage when more physical servers running NCs are added to the cluster as worker nodes.

Finally, Eucalyptus provides a VMware Broker enabling deployment of VMs on VMware clusters, commonly found on institutional
infrastructures. The  VMware Broker works by mediating all interactions between the CC and VMware hypervisors layer on the cluster, essentially
treating the hypervisor as a NC that is passed instructions to start or stop VMs given by user application through the cloud API. \pb


\subsubsection*{OpenStack}
Enis and collaborators. Describe the OpenStack architecture and components (Nova, S3 etc) that it provides.
Example private cloud installations that use it. \pb


%%%%%%%%%%%%%%%%%%%%%%%%%%%%
%% Results and Discussion %%
%%

\section*{Results and Discussion}


\subsection*{Porting the CloudBioLinux-CloudMan Framework to Private Clouds} 

\subsubsection*{The CloudMan Implementation For Scalable Computing on The Cloud}

Enis Afgan: technical description of CloudMan as it stands for Amazon EC2. For example it uses two data volumes,
pulls the code on boot from S3, how does it work with snapshots etc. This will serve as leeway for when describing
how we mapped these components to the other platforms, see below. \pb

\subsubsection*{Porting CloudBioLinux-CloudMan to Eucalyptus}
Alex, technical description of changes to the CloudMan EC2 codebase to run on Eucalyptus. How components that
CloudMan uses on EC2 where mapped to Eucalyptus. Specific nagging that Eucalyptus does with running CloudMan -
i.e. different responses from the API returned when doing calls via boto, and changes in the cloudman codebase in
order to take care of these. Also is there something that sysadmins that set up Eucalyptus should pay attention too 
(in the settings, in order to leverage those applications that create scriptable infrastructures by calling the API) \pb

While adapting Galaxy-Cloudman for use on the Eucalyptus private cloud, we ran into problems common to migrating 
between systems that both track an evolving standard; some features of AWS are not implemented in Eucalyptus, and some 
have differences in implementation. The primary absence was the ability to add arbitrary tags (key-value pairs) to running instances 
(virtual machines) or volumes (virtual disk drives). The primary difference in functionality is that AWS uses Xen virtualization, 
while Eucalyptus uses KVM virtualization. This meant that, while AWS can attach a new volume to an instance at an arbitrary device 
id (e.g. /dev/xvdf), Eucalyptus will always attach at the next available device id (e.g. /dev/vdb), regardless of what id was specified 
in the attachment request. We were forced to write code to emulate tagging and detect attachment device ids within the Cloudman 
infrastructure. 

\subsubsection*{Porting CloudBioLinux-CloudMan to OpenStack}
Enis Afgan: technical description of changes to the CloudMan EC2 codebase to run on OpenStack.  Changes that have be
made to code in order to communicate with the API correctly. \pb

\subsection*{Execution and Performance of Bioinformatic Pipelines on Private Clouds} 

\subsubsection*{An HMM-BLAST Pipeline for Metagenomic Annotation at JCVI}
Ntino... I got the numbers, need to write it. \pb

\subsubsection*{A Variant Calling Analysis Pipeline at HSPH}
\pb

\subsubsection*{Enis Other Example of Pipeline}
\pb

%%%%%%%%%%%%%%%%%%%%%%
\section*{Conclusions}  \ldots


\bigskip

%%%%%%%%%%%%%%%%%%%%%%%%%%%%%%%%
\section*{Author's contributions}
    Alex Richter and Konstantinos Krampis worked on the technical aspects of porting CloudMan
    to the Eucalyptus Cloud. Enis Afgan and .... worked on porting CloudMan on the Openstack
    Cloud. Brad Chapman implemented the parallel sequence data analysis pipelines used for benchmarking
    the two different Clouds.

    

%%%%%%%%%%%%%%%%%%%%%%%%%%%
\section*{Acknowledgements}
  \ifthenelse{\boolean{publ}}{\small}{}
  Text for this section \ldots
 
%%%%%%%%%%%%%%%%%%%%%%%%%%%%%%%%%%%%%%%%%%%%%%%%%%%%%%%%%%%%%
%%                  The Bibliography                       %%
%%                                                         %%              
%%  Bmc_article.bst  will be used to                       %%
%%  create a .BBL file for submission, which includes      %%
%%  XML structured for BMC.                                %%
%%  After submission of the .TEX file,                     %%
%%  you will be prompted to submit your .BBL file.         %%
%%                                                         %%
%%                                                         %%
%%  Note that the displayed Bibliography will not          %% 
%%  necessarily be rendered by Latex exactly as specified  %%
%%  in the online Instructions for Authors.                %% 
%%                                                         %%
%%%%%%%%%%%%%%%%%%%%%%%%%%%%%%%%%%%%%%%%%%%%%%%%%%%%%%%%%%%%%

\newpage
{\ifthenelse{\boolean{publ}}{\footnotesize}{\small}
 \bibliographystyle{bmc_article}  % Style BST file
  \bibliography{bmc_article} }     % Bibliography file (usually '*.bib' ) 

%%%%%%%%%%%

\ifthenelse{\boolean{publ}}{\end{multicols}}{}







\end{bmcformat}
\end{document}







